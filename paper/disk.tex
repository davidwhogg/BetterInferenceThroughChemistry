% This file is part of the BetterInferenceThroughChemistry project.
% Copyright 2018 the authors.

% To-dos
% ------
% - draft introduction
% - draft method
% - draft data & results
% - draft discussion
% - look at element wrongness with Teff, logg; and say words about it; this could KILL the method.
% - Bovy point that selection can enter at higher order bc luminosity

% Style notes
% - 

\documentclass[modern]{aastex62}
\usepackage{graphicx}
\usepackage{xcolor}
\usepackage[sort&compress]{natbib}
\usepackage[hang,flushmargin]{footmisc}

% units macros
\newcommand{\unit}[1]{\mathrm{#1}}
\newcommand{\km}{\unit{km}}
\newcommand{\s}{\unit{s}}
\newcommand{\kms}{\km\,\s^{-1}}
\newcommand{\pc}{\unit{pc}}
\newcommand{\kpc}{\unit{kpc}}

% math macros
\newcommand{\dd}{\mathrm{d}}
\newcommand{\T}{^{\mathsf{T}}}

% chemical macros
\newcommand{\alphafe}{[\alpha/\mathrm{Fe}]}

% text macros
\newcommand{\documentname}{\textsl{Article}}
\newcommand{\sectionname}{Section}
\newcommand{\todo}[1]{\textcolor{red}{#1}}  % gotta have \usepackage{xcolor} in main doc or this won't work
\newcommand{\acronym}[1]{{\small{#1}}}
\newcommand{\project}[1]{\textsl{#1}}
\newcommand{\foreign}[1]{\textsl{#1}}
\newcommand{\galah}{\project{\acronym{GALAH}}}
\newcommand{\gaia}{\project{Gaia}}

\setlength{\parindent}{1.4em} % trust in Hogg
\shorttitle{chemical tangents}
\shortauthors{hogg-n-price-whelan}

\begin{document}\sloppy\sloppypar\raggedbottom\frenchspacing % trust in Hogg
\graphicspath{ {figures/} }
\DeclareGraphicsExtensions{.pdf,.eps,.png}

\title{\textbf{%
Chemical-tangents method:\\
Identifying orbits in phase space through element-abundance invariances}}

\author[0000-0003-2866-9403]{David W. Hogg}
\affil{Center for Computational Astrophysics, Flatiron Institute, 162 Fifth Ave, New York, NY 10010, USA}
\affil{Center for Cosmology and Particle Physics, Department of Physics, New York University, 726 Broadway, New York, NY 10003, USA}
\affil{Center for Data Science, New York University, 60 Fifth Ave, New York, NY 10011, USA}
\affil{Max-Planck-Institut f\"ur Astronomie, K\"onigstuhl 17, D-69117 Heidelberg}

\author[0000-0003-0872-7098]{Adrian~M.~Price-Whelan}
\affiliation{Department of Astrophysical Sciences,
             Princeton University, Princeton, NJ 08544, USA}

\author{Others}

\begin{abstract}\noindent
% context
Some stellar surface element abundances are invariants: They don't change with time.
In this way they are like dynamical actions that are invariant along a stellar trajectory.
% aims
Here we capitalize on the invariance of element abundances to identify dynamical
invariants, by using the fact that---for a phase-mixed population---the abundance
distribution ought to depend only on dynamical invariants, but not at all on time-variable
dynamical phases (conjugate actions).
That is, the stellar orbits should be tangent everywhere to the
element-abundance level hyper-surfaces or iso-abundance contours in 6-d phase
space, or orthogonal to the 6-d abundance gradients.
For a first project, we consider only the vertical structure and dynamics of the Milky Way
disk and only a few abundances, to demonstrate feasibility.
% methods
We make use of \galah\ data on red-giant stars in the Solar Neighborhood.
We parameterize the vertical density structure of the disk with a central density, an
exponential scale height, and a vertical location of the mid-plane.
We fit the abundance distribution as a Gaussian with a mean and variance that depend
on dynamical invariants computed from the parameterized density structure.
% results
We find FOO WHATEVER.
These chemical-tangents methods have great advantages over other methods
for inferring dynamical actions or orbits: They are not sensitive to spectroscopic
selection functions, which have aspects that can't be known precisely; they
do not require taking second moments of velocity data, which are not usually measured
well; and they do not require a full forward model of the distribution function,
which is hard to model or marginalize with sufficient flexibility.
\end{abstract}

\keywords{
  foo
  ---
  bar
}

\section{Introduction}

Most old-school inferences about the gravitational potential of the
Milky Way, or, equivalently, the gravitational acceleration field, or,
equivalently, the mass density, are made by assuming that stars are
phase-mixed on regular orbits.
This assumption is valuable, because it permits inference of the
dynamics from the current phase-space snapshot, and it leads to a
simplification in which the distribution function can depend only on
dynamical invariants, such as the dynamical actions.
Within this assumption-context, there are several methods available to the
contemporary astrophysical dynamicist:
One is through \emph{moment relationships}, in which the Jeans equation (relating second moments
of stellar velocities) connects mean products of velocities by a mixture of potential and
tracer-density derivatives.
Jeans modeling has the great advantage that---because it depends only on velocity
moments---a survey selected in position-space can be used without much
(or maybe any) sensitivity to the (usually complex) survey selection
function.
It has the disadvantage that it inovlves second moments of the velocity
field, which are noisily measured in any finite data set, and also a lossy
compression of the relevant data.
Another method is full \emph{forward modeling} of the position and velocity
data, with both the dynamics and the phase-space distribution function
(which, by assumption, depends only on dynamical invariants) and the
force law or potential as free functions.
This has the advantage that it exploits all of kinematic information in
the data set.
But it has the great disadvantages of requiring an accurate selection function
and requiring optimization or marginalization of very flexible models for
the distribution function, which is not just technically challenging (CITE BOVY HOGG), but it
also puts the computational load on the part of the problem that is (usually)
uninteresting.
There are intermediate methods as well; in which some form of forward
modeling is applied, but it is performed conditionally on position to reduce
the sensitivity to the selection function (CITE).
But suffice it to say that the old-school methods either strongly reduce the
data through moments, or else require large computational efforts.

A great deal has changed recently.
There are now million-star spectroscopic surveys that are returning not
just kinematics but multi-element chemical abundances (HOGG CITE).
These surveys have many motivations, but among them is the hope for
chemical tagging (HOGG CITE), in which the (nearly) invariant surface
abundances observable in a stellar atmosphere are used as tags that identify
pairs or groups of stars that were born in the same molecular cloud at the same time,
even if that time was far in the past.
Because the chemical tagging approach capitalizes on element-abundance invariants
rather than dynamical invariants, it could potentially permit dynamical inferences
about the Milky Way that do not rely on the phase-mixed assumption underlying
most old-school methods.
Chemical tagging has not yet successfully revealed the origins of
the Milky Way, but element-abundance tags are starting to be used to identify stars that have
migrated far from their birth locations (CITE).
In its most na\"ive or strongest form, chemical tagging works when high-dimensional
element-abundance measurements create such a specific fingerprint for a star that it
can only have one (or a few) possible birthplace events in the 3+1-dimensional past.
That's an assumption, and one that is increasingly untenable (CITE).
It is time to look at either softer versions of the chemical-tagging goal, or else
look at other ways of exploiting the near-invariance of surface abundances.

The project reported here has the latter goal:
We are going to use element-abundance invariances as a tool to discover dynamical
invariants.
Like the old-school methods we opened with, we will make the well-mixed assumption.
But unlike the chemical-tagging approaches, we will not have to model or identify
the origins of the stars we use.

In a well-mixed population, stars are in a kinematic steady state:
As time goes on, stars move along their orbits, but they don't change their dynamical
invariants (by assumption!) and every location in angle (and by ``angle'' we mean the
coordinate conjugate to action in action--angle coordinates) is equally likely, or
equally probably populated in any snapshot.
The chemical-tagging insight is that the stars also don't change their surface element
abundances as they orbit.
That is, in a well-mixed galaxy, the element abundances and the dynamical actions have
something in common:
They don't change with time, while the conjugate angles do.
This means that, for a well-mixed population, the detailed element abundances can only
be a function of actions, and never a function of conjugate angles.
That's a remarkably informative constraint on the configuration of the Milky Way in
the space of positions (3), velocities (3), and detailed abundances (10 to 30, depending
on survey).

Let's get more specific.
Consider a collection of stars (localized, say, in phase space) for which we have
measured $D$ element abundances.
This collection of stars is drawn from some density or distribution in
the $D$-dimensional element-abundance space.
Now consider how this distribution varies as we move around in phase space.
It will vary in general (there are radial and vertical abundance gradients in the
disk, for example).
Any gradient we observe in these element-abundance distributions with respect to
phase-space coordinates (that is, the six-gradient) must not project
onto the directions of increasing (or decreasing) conjugate angles in phase space.
That is, all gradients with respect to phase-space coordinates
of the element-abundance distribution must be orthogonal to the
directions of increase (or decrease) of the conjugate angles, and lie in the subspace
of the directions of increase (or decrease) of the dynamical actions.
The trajectories of stars in the phase space (the dynamical tori) must lie along or
describe level surfaces in the element-abundance distribution!

One of the crazy things about all this is that any description of a general distribution in $D$-space
requires a \emph{lot of parameters}.
Even a trivial distribution---the Normal distribution---requires $0.5\,D\,(D+3)$ parameters for its
complete description, and anything more complex has more.
That means that there are a very large set of element-abundance statistics for which the
six-gradient exists or could be constraining for this project.
Not all of these parameters can be reliably measured in a finite data set,
let alone reliably seen to vary with phase-space location.
However,
we have discovered, in effect, a \emph{combinatorically large set of constraints on the
dynamical actions and conjugate angles}.

HOGG Aside: Controversy over the massive/dense disk. How we might resolve it.
But still making the phase-mixed assumption!

\section{Assumptions}

The method and results in this \documentname\ will flow from a specific set of
assumptions.
That is, we are going to make hard (and sometimes controversial) assumptions,
and although the assumptions will be questionable, the claim will be that the
method is correct under the assumptions.
\begin{description}
\item[well mixed] HOGG: All angles equally likely. There is probably also a regular-orbits
assumption here, but it won't enter in this \documentname\ because we are only doing
one-dimensional dynamics.

\item[selection] HOGG: Selection depends on position in the Milky Way, but not
on element abundances. Really it could even depend on velocity! But it can't depend
on abundances. That might be very slightly wrong.

\item[kinematic measurements] HOGG: Very strong assumption that phase space positions
in 6-d are accurate and precise. So precise we don't have to generate them, we can
condition on them!

\item[chemical measurements] HOGG: Assumption that stars in different parts of phase
space get the same abundance measurements. Questionable in certain kinds of samples.
HOGG: Oddly I don't think we need to assume that the chemical measurements have good
(or any) noise estimates. They could be terrible, I think.
\end{description}

\section{Method}

In general the idea underlying this method is to figure out what mass-density distribution
in the Galaxy leads to orbits in phase space such that those orbits are level
hyper-surfaces in phase space of statistics of the chemical-abundance distribution.
Alternatively, the idea is to figure out what mass-density distribution leads to orbits
on which statistics of the chemical-abundance distribution do not vary with conjugate
angle.
And of course this will all be done probabilistically, inasmuch as that is easy.
So the idea is to perform probabilistic inference of the mass-density distribution.

HOGG: Dropping from 3-d to 1-d. In this case, let's start with 1-d. What about that is simpler?

HOGG: maximize conjugate-angle diversity at fixed element abundances?
Or minimize abundance-ratio diversity at fixed orbit? The latter! Why?

\section{Data and results}

\begin{figure}
\caption{The data used in this \documentname. Each panel shows the stellar points in the
plane of vertical position $z$ and vertical velocity $v_z$, colored by a different element
abundance ratio. These are phase-space plots, but colored by the abundance ratios.
It is clear in these plots that the abundance ratio level surfaces will be
constraining on the orbits.\label{fig:data}}
\end{figure}

\begin{figure}
\caption{Abundances versus conjugate angles, for different mass distributions.
Each panel shows an abundance ratio HOGG vs conjugate angle $\phi_z$, for three different
choices of the mean mass density at the disk midplane, for fixed scale height and midplane
location. Note that when the mass distribution gets very wrong, the stars show strong
dependences of abundance ratio on conjugate angle.\label{fig:density}}
\end{figure}

\begin{figure}
\caption{Same as \figurename~\ref{fig:density}, but varying the scale height. Note that
varying the scale height leads to a higher order variation with angle.\label{fig:scaleheight}}
\end{figure}

\begin{figure}
\caption{Same as \figurename~\ref{fig:density}, but varying the midplane location. Note
that varying the midplane location leads to a lower order variation with angle.\label{fig:midplane}}
\end{figure}

\begin{figure}
\caption{Posterior samples for the model parameters.\label{fig:samples}}
\end{figure}

\begin{figure}
\caption{Posterior orbit structure. Same as \figurename~\ref{fig:data} but now with
posterior-sensible orbits shown to indicate that the orbits do come close to describing
level surfaces in the phase space.}
\end{figure}

\section{Discussion}

\acknowledgements
It is a pleasure to thank
  Jo Bovy (Toronto), 
  Melissa Ness (Columbia),
  and
  Hans-Walter Rix (\acronym{MPIA})
for valuable discussions that made this project possible.

HOGG: \galah\ ack??

HOGG: \gaia\ ack.

\software{the usual}

\facilities{the usual}

\end{document}
