% This file is part of the BetterInferenceThroughChemistry project.
% Copyright 2018 the authors.

% To-dos
% - draft introduction
% - draft method
% - draft data & results
% - draft discussion

% Style notes
% - 

\documentclass[modern]{aastex62}
\usepackage{graphicx}
\usepackage{xcolor}
\usepackage[sort&compress]{natbib}
\usepackage[hang,flushmargin]{footmisc}

% units macros
\newcommand{\unit}[1]{\mathrm{#1}}
\newcommand{\km}{\unit{km}}
\newcommand{\s}{\unit{s}}
\newcommand{\kms}{\km\,\s^{-1}}
\newcommand{\pc}{\unit{pc}}
\newcommand{\kpc}{\unit{kpc}}

% math macros
\newcommand{\dd}{\mathrm{d}}
\newcommand{\T}{^{\mathsf{T}}}

% chemical macros
\newcommand{\alphafe}{[\alpha/\mathrm{Fe}]}

% text macros
\newcommand{\documentname}{\textsl{Article}}
\newcommand{\sectionname}{Section}
\newcommand{\todo}[1]{\textcolor{red}{#1}}  % gotta have \usepackage{xcolor} in main doc or this won't work
\newcommand{\acronym}[1]{{\small{#1}}}
\newcommand{\project}[1]{\textsl{#1}}
\newcommand{\foreign}[1]{\textsl{#1}}
\newcommand{\galah}{\project{\acronym{GALAH}}}
\newcommand{\gaia}{\project{Gaia}}

\setlength{\parindent}{1.4em} % trust in Hogg
\shorttitle{chemical tangents}
\shortauthors{hogg-n-price-whelan}

\begin{document}\sloppy\sloppypar\raggedbottom\frenchspacing % trust in Hogg
\graphicspath{ {figures/} }
\DeclareGraphicsExtensions{.pdf,.eps,.png}

\title{\textbf{Chemical-tangents method:\\
               Identifying the vertical density structure of the Milky Way disk
               through element-abundance invariances}}

\author[0000-0003-2866-9403]{David W. Hogg}
\affil{Center for Computational Astrophysics, Flatiron Institute, 162 Fifth Ave, New York, NY 10010, USA}
\affil{Center for Cosmology and Particle Physics, Department of Physics, New York University, 726 Broadway, New York, NY 10003, USA}
\affil{Center for Data Science, New York University, 60 Fifth Ave, New York, NY 10011, USA}
\affil{Max-Planck-Institut f\"ur Astronomie, K\"onigstuhl 17, D-69117 Heidelberg}

\author[0000-0003-0872-7098]{Adrian~M.~Price-Whelan}
\affiliation{Department of Astrophysical Sciences,
             Princeton University, Princeton, NJ 08544, USA}

\author{Others}

\begin{abstract}\noindent
% context
Some stellar surface element abundances are invariants: They don't change with time.
In this way they are like dynamical actions that are invariant along a stellar trajectory.
% aims
Here we capitalize on the invariance of element abundances to identify dynamical
invariants, by using the fact that---for a phase-mixed population---the abundance
distribution ought to depend only on dynamical invariants, but not at all on time-variable
dynamical phases.
That is, the stellar orbits should be tangent everywhere to the
element-abundance level surfaces or iso-abundance contours in phase
space, or orthogonal to the chemical gradients.
For a first project, we consider only the vertical structure and dynamics of the Milky Way
disk and only $\alphafe$ abundances, to demonstrate feasibility.
% methodse
We make use of \galah\ data on red-giant stars in the Solar Neighborhood.
We parameterize the vertical density structure of the disk as BLAH BLAH.
We fit the $\alphafe$ distribution as a Gaussian with mean and variance that depends
smoothly on rdynamical invariants computed from the density structure.
% results
We find FOO WHATEVER.
These chemical-tangents methods have great advantages over other methods
for inferring dynamical actions or orbits: They are not sensitive to spectroscopic
selection functions, which have aspects that can't be known precisely; they
do not require taking second moments of velocity data, which are not usually measured
well; and the do not require a full forward model of the distribution function,
which is hard to model or marginalize with sufficient flexibility.
\end{abstract}

\keywords{
  foo
  ---
  bar
}

\section{Introduction}

Most old-school inferences about the gravitational potential of the
Milky Way, or, equivalently, the gravitational acceleration field, or,
equivalently, the mass density, are made by assuming that stars are
phase-mixed on regular orbits.

This assumption is valuable, because it permits inference of the
dynamics from the current phase-space snapshot, and it leads to a
simplification in which the distribution function can depend only on
dynamical invariants, such as the dynamical actions.

Within this assumption-context, there are many methods available to the
contemporary astrophysical dynamicist:

One is \emph{Jeans modeling}, in which relationships among second moments
of stellar velocities are determined by a mixture of potential and
tracer-density derivatives.

This has the great advantage that because it depends only on velocity
moments, a survey selected in position-space can be used without much
(or maybe any) sensitivity to the (usually complex) survey selection
function.

Another method is full 



Within the 

How you do inferences for phase-mixed populations.

How all these inferences are either high-order (like Jeans) or hard (like forward modeling).

The element-abundance-invariance insight.

Aside: Controversy over the massive/dense disk.

\section{Method}

In general, in 3-d, what is the idea?

In this case, let's start with 1-d. What about that is simpler?

\section{Data and results}

\section{Discussion}

\acknowledgements
It is a pleasure to thank
  Melissa Ness (Columbia),
  and
  Hans-Walter Rix (\acronym{MPIA})
for valuable discussions that made this project possible.

HOGG: \galah\ ack??

HOGG: \gaia\ ack.

\end{document}
