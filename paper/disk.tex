% This file is part of the BetterInferenceThroughChemistry project.
% Copyright 2018 the authors.

% To-dos
% - 

% Style notes
% - 

\documentclass[modern]{aastex62}
\usepackage{graphicx}
\usepackage{xcolor}
\usepackage[sort&compress]{natbib}
\usepackage[hang,flushmargin]{footmisc}

% units macros
\newcommand{\unit}[1]{\mathrm{#1}}
\newcommand{\km}{\unit{km}}
\newcommand{\s}{\unit{s}}
\newcommand{\kms}{\km\,\s^{-1}}
\newcommand{\pc}{\unit{pc}}
\newcommand{\kpc}{\unit{kpc}}

% math macros
\newcommand{\dd}{\mathrm{d}}
\newcommand{\T}{^{\mathsf{T}}}

% text macros
\newcommand{\documentname}{\textsl{Article}}
\newcommand{\sectionname}{Section}
\newcommand{\todo}[1]{\textcolor{red}{#1}}  % gotta have \usepackage{xcolor} in main doc or this won't work
\newcommand{\acronym}[1]{{\small{#1}}}
\newcommand{\project}[1]{\textsl{#1}}
\newcommand{\foreign}[1]{\textsl{#1}}
\newcommand{\HARPS}{\project{\acronym{HARPS}}}
\newcommand{\HIRES}{\project{\acronym{HIRES}}}
\newcommand{\RV}{\acronym{RV}}
\newcommand{\CRLB}{\acronym{CRLB}}

\setlength{\parindent}{1.4em} % trust in Hogg
\shorttitle{element abundances ftw}
\shortauthors{hogg}

\begin{document}\sloppy\sloppypar\raggedbottom\frenchspacing % trust in Hogg
\graphicspath{ {figures/} }
\DeclareGraphicsExtensions{.pdf,.eps,.png}

\title{\textbf{Identifying dynamical invariants by chemical invariance: \\
               The vertical density structure of the Milky Way disk}}

\author[0000-0003-2866-9403]{David W. Hogg}
\affil{Center for Computational Astrophysics, Flatiron Institute, 162 Fifth Ave, New York, NY 10010, USA}
\affil{Center for Cosmology and Particle Physics, Department of Physics, New York University, 726 Broadway, New York, NY 10003, USA}
\affil{Center for Data Science, New York University, 60 Fifth Ave, New York, NY 10011, USA}
\affil{Max-Planck-Institut f\"ur Astronomie, K\"onigstuhl 17, D-69117 Heidelberg}

\author[0000-0003-0872-7098]{Adrian~M.~Price-Whelan}
\affiliation{Department of Astrophysical Sciences,
             Princeton University, Princeton, NJ 08544, USA}

\author{Others}

\begin{abstract}\noindent
% context
Some stellar surface element abundances are invariants: They don't change with time.
In this way they are like dynamical actions that are invariant along a stellar trajectory.
% aims
Here we capitalize on the invariance of element abundances to identify dynamical
invariants, by using the fact that---for a phase-mixed population---the abundance
distribution ought to depend only on dynamical invariants, but not at all on time-variable
dynamical phases.
For a first project, we consider only the vertical structure and dynamics of the Milky-Way
disk and only \alphafe\ abundances, to demonstrate feasibility.
% methods
We make use of \GALAH\ data on red-giant stars in the Solar Neighborhood.
...
% results
...
One great advantage of these methods is that they do not depend on spectroscopic
selection functions, which have aspects that can't be known precisely.
\end{abstract}

\keywords{
  foo
  ---
  bar
}

\section{Introduction}

Hello world!

\section{Method}

\section{Data and results}

\section{Discussion}

\end{document}
